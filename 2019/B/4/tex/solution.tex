\documentclass{article}
\usepackage{geometry}
\usepackage{graphicx}
\graphicspath{ {./} }

\geometry{margin=0.75in}


\begin{document}

\title{Riešenie 4. úlohy kategórie B}
\author{Jozef Komáromy}

\maketitle

\section{Podúloha A}

\begin{figure}[h]
  \centering
\includegraphics{A}
\end{figure}

\section{Podúloha B}

Ak je postavený najväčší možný počet ciest (\(n + 2 * (n - 3)\)), Zuzanka sa z akéhokoľvek kúsku lúky dokáže dostať vždy presne do 3 domov. Je to tak, pretože maximalizovaním počtu ciest vznikajú časti lúky, ktorých odvod tvoria presne tri prepojené domy.

\section{Podúloha C}

Vzťah medzi celkovým počtom ciest \(n\) a počtom častí na ktoré by bola lúka rozdelená \(p\) by bol nasledovný: \(p = 2 + 2 * (n - 3)\). Je to tak pretože uzavretý obvod kruhu delí lúku na 2 časti a každá nasledovne pridaná cesta oddelí jednu ďalšiu časť lúky.

\section{Podúloha D}

Nespĺňajú vzťah \(p = 2 + 2 * (n - 3)\) pretože všetky domy nie sú sériovo prepojené (nevytvárajú \(n\)-uholník, ktorým by delili lúku na 2 časti) \(n\) cestami. Ak by sme prepojily ktorékoľvek dva domy z prvej skupiny cestou, počet častí lúky by sa navýšil o 1. Ak by sme ale prepojily ľubovolný dom s domom z druhej alebo tretej skupiny, počet častí lúky by sa nezmenil. (V oboch prípadoch je predpoklad že by sme pri spájaní domov cestou dodržali pravidlá uvedené v zadaní podúlohy A)

\section{Podúloha E}

Najviac sa dá postaviť \(n + 2 * (n - 3)\) ciest. Prvých \(n\) ciest bude tvoriť uzavrétý obvod kruhu na ktorom sú postavené domy. Ďalších \(n - 3\) ciest bude viesť z ľubovolného domu s poradovým číslom \(d\) do domov s ktorými nie je spojený, tieto cesty sa budú nachádzať vo vnútri pôvodného kruhu domov. Posledných \(n - 3\) ciest bude viesť z domu s poradovým číslom \(d + 1\) do domov s ktorými nie je spojený, tieto cesty sa budú nachádzať na vonkajšej strane kruhu.

\end{document}

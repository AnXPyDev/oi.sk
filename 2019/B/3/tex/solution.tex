\documentclass{article}
\usepackage{geometry}
\usepackage{listings}

\geometry{margin=0.75in}


\begin{document}

\title{Riešenie 3. úlohy kategórie B}
\author{Jozef Komáromy}

\maketitle

\lstset{
  numberstyle=\small,
  frame=single,
  language={[5.0]Lua}
}

\section{Súťažná úloha}

Tento program je zapísaný v jazyku Lua, \textbf{v ktorom sa indexy množín začínajú od 1}.


\subsection{Inicializácia premenných podľa zadania}

\begin{lstlisting}
local n, urovne, u, k, q, pohyb, mozne_ukoncenia

n = 5

urovne = {
  {-1},
  {1, 1, 1},
  {1, 1, 2, 3, 3, 3, 3},
  {1, 1, 3, 3, 3, 5},
  {2, 2}
}

u = 2 + 1
k = 5
q = 7

pohyb = {"H", "D", "D", "H", "H", "H", "D"}
mozne_ukoncenia = {}
\end{lstlisting}

\subsection{Funkcia koniec}


Táto funkcia pridá do množiny \(mozne\_zakoncenia\) bod \({u, k}\), ak ho už množina neobsahuje.


\begin{lstlisting}
function koniec(u, k)
  for i, ukoncenie in ipairs(mozne_ukoncenia) do
    if ukoncenie[1] == u and ukoncenie[2] == k then
      return
    end
  end
  table.insert(mozne_ukoncenia, {u, k})
end
\end{lstlisting}

\subsection{Funkcia spocitaj\_moznosti}

Táto funkcia je rekurzívna, a je spustená pre každú komôrku, do ktorej sa mravec môže dostať. Ak sa dostane na posledný krok, zavolá funkciu \(koniec\) pre určitú pozíciu \(u, k\).
Ak sa mravec v \(n\)-tom kroku pohol smerom hore (\(pohyb[krok] == ``H''\)), funkcia sa spustí rekurzívne pre komôrku s poradovým číslom \(urovne[u][k]\) na úrovni \(u - 1\) (tú, na ktorú ukazuje komôrka \(u, k\)).
Naopak, ak sa mravec pohne smerom dolu (\(pohyb[krok] == ``D''\)), funkcia sa spustí pre každú komôrku na úrovni \(u + 1\), ktorá ukazuje na komôrku \(u, k\) (\(urovne[u+1][j] == k\)).
Pri každom rekurzívnom spustení funkcie sa navýši premenná \(krok\) o \(1\).

\begin{lstlisting}
function spocitaj_moznosti(u, k, krok)
  if krok > q then
    koniec(u, k)
    return
  end
  if pohyb[krok] == "H" then
    spocitaj_moznosti(u - 1, urovne[u][k], krok + 1)
  elseif pohyb[krok] == "D" then
    if u + 1 > n then
      return
    end
    for j, _ in ipairs(urovne[u + 1]) do
      if urovne[u + 1][j] == k then
        spocitaj_moznosti(u + 1, j, krok + 1)
      end
    end
  end
end

\end{lstlisting}

\subsection{Zvyšok programu}

Funkcia \(spocitaj\_moznosti\) je spustená pre počiatočnú komôrku \(u, k\) a vypíše veľkosť množiny \(mozne\_zakoncenia\) (počet rôznych možných zakončení).

\begin{lstlisting}[language={[5.0]Lua}]
spocitaj_moznosti(u, k, 1)
print(#mozne_ukoncenia)
\end{lstlisting}

\end{document}

\documentclass{article}
\usepackage{geometry}
\usepackage{listings}
\usepackage{amsmath}
\usepackage{graphicx}

\graphicspath{ {../images/} }

\geometry{margin=0.75in}


\begin{document}

\title{Riešenie 4. úlohy kategórie A}
\author{Jozef Komáromy}

\maketitle

\lstset{
  numberstyle=\small,
  frame=single,
  language={Python}
}

Všetky programy sú zapísané v jazyku Python s predpokladom čísla $n$ uloženého v premennej $N$. V programoch sú použité nedefinované funkcie $Write$ a $Read$ rovnako ako v študijnom text v zadaní.

\section*{Podúloha A}

Všetky prvky vypíšeme po riadkoch jednoduchou iteráciou súradnice $y$ (poradie riadku, začína nulou) od $0$ po $N-1$ a iteráciou súradnice $x$ (poradie stĺpca, začína nulou) od $0$ po $N-1$ pre každú súradnicu $y$. Pre každý prvok vypočítame jeho poradie na disku vzťahom $i = y * N + x$. Nasledovne toto poradie použijeme ako argument pre funkciu $Read$ aby sme získali v poradí $i$-tý blok dát z disku ($blok = Read(i)$). Tento blok vypíšeme funkciou $print$ (\(print(blok)\)). Aby funkcia $print$ neukončila riadok a vypísala medzeru medzi prvkami, použijeme argument \(end = " "\). Celý tento proces spojíme do jedného riadku ako \(print(Read(y * N + x), end = "\ ")\). Po vypísaní každého prvku v jednom riadku (po konci vnútorného for cyklu) ukončíme riadok.

\begin{lstlisting}
for y in range(N):
  for x in range(N):
    print(Read(y * N + x), end = " ")
  print("", end = "\n")
\end{lstlisting}

\section*{Podúloha B}

Riešenie tejto podúlohy bude podobné ako riešenie podúlohy A. Narozdiel od podúlohy A budeme vypisovať stĺpce ako riadky. Aby sme toto dosiahly, stačí len zameniť vonkajší cyklus za vnútorný tak aby sme pre každú súradnicu $x$ vypísali všetky prvky od $y = 0$ až po $y = N-1$. Vzťah pre výpočet poradia zostane rovnaký.

\begin{lstlisting}
for x in range(N):
  for y in range(N):
    print(Read(y * N + x), end = " ")
  print("", end = "\n")
\end{lstlisting}

\end{document}

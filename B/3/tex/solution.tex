\documentclass{article}
\usepackage{geometry}
\geometry{margin=1in}


\begin{document}

\title{Riešenie 3. úlohy kategórie B}
\author{Jozef Komáromy}

\maketitle

\section{Podmienky pre funkciu algoritmu}

Po načítaní súboru si všetky slová zoradíme do listu.
Taktiež potrebujeme premennú so všetkými písmenamy abecedy v stringu.

\section{Hlavný algoritmus}
\subsection{Cyklus}
Pre každé slovo od predposledného slova porovnávame jeho jednotlivé písmená so slovom o jedno miesto za ním.
\subsection{Porovnávanie}
Ak nájdeme nezhodu v písmenách slov na tom istom indexe stringu, tak v premennej abecedy presunieme nezhodné písmeno z teraz porovnávaného slova pred na index nezhodného písmena zo slova ku ktorému porovnávame, potom pokračujeme na ďalšie slovo.

Ak nenájdeme nezhodu a dostaneme sa ku koncu jedného zo slov, nemeníme poradie abecedy, ale pokračujeme na ďalší pár slov.

\section{Ukončenie programu}
Po ukončení algoritmu vytlačíme premennú abecedy.

\end{document}
